% Options for packages loaded elsewhere
\PassOptionsToPackage{unicode}{hyperref}
\PassOptionsToPackage{hyphens}{url}
%
\documentclass[
]{article}
\usepackage{amsmath,amssymb}
\usepackage{iftex}
\ifPDFTeX
  \usepackage[T1]{fontenc}
  \usepackage[utf8]{inputenc}
  \usepackage{textcomp} % provide euro and other symbols
\else % if luatex or xetex
  \usepackage{unicode-math} % this also loads fontspec
  \defaultfontfeatures{Scale=MatchLowercase}
  \defaultfontfeatures[\rmfamily]{Ligatures=TeX,Scale=1}
\fi
\usepackage{lmodern}
\ifPDFTeX\else
  % xetex/luatex font selection
\fi
% Use upquote if available, for straight quotes in verbatim environments
\IfFileExists{upquote.sty}{\usepackage{upquote}}{}
\IfFileExists{microtype.sty}{% use microtype if available
  \usepackage[]{microtype}
  \UseMicrotypeSet[protrusion]{basicmath} % disable protrusion for tt fonts
}{}
\makeatletter
\@ifundefined{KOMAClassName}{% if non-KOMA class
  \IfFileExists{parskip.sty}{%
    \usepackage{parskip}
  }{% else
    \setlength{\parindent}{0pt}
    \setlength{\parskip}{6pt plus 2pt minus 1pt}}
}{% if KOMA class
  \KOMAoptions{parskip=half}}
\makeatother
\usepackage{xcolor}
\usepackage[margin=1in]{geometry}
\usepackage{longtable,booktabs,array}
\usepackage{calc} % for calculating minipage widths
% Correct order of tables after \paragraph or \subparagraph
\usepackage{etoolbox}
\makeatletter
\patchcmd\longtable{\par}{\if@noskipsec\mbox{}\fi\par}{}{}
\makeatother
% Allow footnotes in longtable head/foot
\IfFileExists{footnotehyper.sty}{\usepackage{footnotehyper}}{\usepackage{footnote}}
\makesavenoteenv{longtable}
\usepackage{graphicx}
\makeatletter
\def\maxwidth{\ifdim\Gin@nat@width>\linewidth\linewidth\else\Gin@nat@width\fi}
\def\maxheight{\ifdim\Gin@nat@height>\textheight\textheight\else\Gin@nat@height\fi}
\makeatother
% Scale images if necessary, so that they will not overflow the page
% margins by default, and it is still possible to overwrite the defaults
% using explicit options in \includegraphics[width, height, ...]{}
\setkeys{Gin}{width=\maxwidth,height=\maxheight,keepaspectratio}
% Set default figure placement to htbp
\makeatletter
\def\fps@figure{htbp}
\makeatother
\ifLuaTeX
  \usepackage{luacolor}
  \usepackage[soul]{lua-ul}
\else
  \usepackage{soul}
\fi
\setlength{\emergencystretch}{3em} % prevent overfull lines
\providecommand{\tightlist}{%
  \setlength{\itemsep}{0pt}\setlength{\parskip}{0pt}}
\setcounter{secnumdepth}{5}
% definitions for citeproc citations
\NewDocumentCommand\citeproctext{}{}
\NewDocumentCommand\citeproc{mm}{%
  \begingroup\def\citeproctext{#2}\cite{#1}\endgroup}
\makeatletter
 % allow citations to break across lines
 \let\@cite@ofmt\@firstofone
 % avoid brackets around text for \cite:
 \def\@biblabel#1{}
 \def\@cite#1#2{{#1\if@tempswa , #2\fi}}
\makeatother
\newlength{\cslhangindent}
\setlength{\cslhangindent}{1.5em}
\newlength{\csllabelwidth}
\setlength{\csllabelwidth}{3em}
\newenvironment{CSLReferences}[2] % #1 hanging-indent, #2 entry-spacing
 {\begin{list}{}{%
  \setlength{\itemindent}{0pt}
  \setlength{\leftmargin}{0pt}
  \setlength{\parsep}{0pt}
  % turn on hanging indent if param 1 is 1
  \ifodd #1
   \setlength{\leftmargin}{\cslhangindent}
   \setlength{\itemindent}{-1\cslhangindent}
  \fi
  % set entry spacing
  \setlength{\itemsep}{#2\baselineskip}}}
 {\end{list}}
\usepackage{calc}
\newcommand{\CSLBlock}[1]{\hfill\break\parbox[t]{\linewidth}{\strut\ignorespaces#1\strut}}
\newcommand{\CSLLeftMargin}[1]{\parbox[t]{\csllabelwidth}{\strut#1\strut}}
\newcommand{\CSLRightInline}[1]{\parbox[t]{\linewidth - \csllabelwidth}{\strut#1\strut}}
\newcommand{\CSLIndent}[1]{\hspace{\cslhangindent}#1}
\pagenumbering{gobble}
\usepackage{booktabs}
\usepackage{longtable}
\usepackage{array}
\usepackage{multirow}
\usepackage{wrapfig}
\usepackage{float}
\usepackage{colortbl}
\usepackage{pdflscape}
\usepackage{tabu}
\usepackage{threeparttable}
\usepackage{threeparttablex}
\usepackage[normalem]{ulem}
\usepackage{makecell}
\usepackage{xcolor}
\usepackage{siunitx}

    \newcolumntype{d}{S[
      table-align-text-before=false,
      table-align-text-after=false,
      input-symbols={-,\*+()}
    ]}
  
\ifLuaTeX
  \usepackage{selnolig}  % disable illegal ligatures
\fi
\usepackage{bookmark}
\IfFileExists{xurl.sty}{\usepackage{xurl}}{} % add URL line breaks if available
\urlstyle{same}
\hypersetup{
  pdftitle={Investigating Vocational Teachers' Informal Workplace Learning Using Experience Sampling},
  pdfauthor={Manuel Böhm},
  hidelinks,
  pdfcreator={LaTeX via pandoc}}

\title{Investigating Vocational Teachers' Informal Workplace Learning
Using Experience Sampling}
\author{Manuel Böhm}
\date{2024-10-30}

\begin{document}
\maketitle

\renewcommand*\contentsname{Article Outline}
{
\setcounter{tocdepth}{2}
\tableofcontents
}
\newpage
\pagenumbering{arabic}

\section*{Acknowledgement}\label{acknowledgement}
\addcontentsline{toc}{section}{Acknowledgement}

We would like to express our gratitude to Miss Julia Banschbach for the
invaluable work in her masters' thesis. The created learning outcome
framework based on the literature as well as her qualitative data
analysis served as a basis for answering the second research question in
this paper.

\newpage

\section*{Abstract}\label{abstract}
\addcontentsline{toc}{section}{Abstract}

\ldots{}

Keywords: workplace learning, teacher training, informal learning,
experience sampling, multilevel modelling

\newpage

\section{Introduction}\label{introduction}

According to Rausch
(\citeproc{ref-rauschUsingDiariesResearch2014}{2014}), \ldots{}

\begin{itemize}
\item
  teacher shortage and difficult working conditions of teachers

  \begin{itemize}
  \item
    stress, coping are important
  \item
    learning of teachers has a particular important role

    \begin{itemize}
    \item
      teachers have to prepare their lessons and
    \item
      furthermore, teachers need to stay up to date
    \item
      The teaching profession has a particular set of characteristics
      and job demands. At the same time, teachers are provided with a
      high degree of freedom or job decision latitude. --\textgreater{}
      Karasek: learning hypothesis
    \end{itemize}
  \item
    vocational schools are under-represented in studies (). while there
    is research tackling other schools, still very little research on
    vocational schools.
  \end{itemize}
\item
  lack of research on teachers at vocational schools ()

  \begin{itemize}
  \tightlist
  \item
    experience sampling
  \end{itemize}
\end{itemize}

Thus, the following research questions will be tackled in this paper:

\begin{enumerate}
\def\labelenumi{\arabic{enumi}.}
\item
  RQ (Stress, coping and learning across activities, control for age,
  sex, jobscope + data from the FBS) - MLM

  Which of teachers' daily work activities are perceived as (a) the most
  stressful and (b) with which of the stressful activities could the
  teachers cope the best? (c) Which of teachers' daily activities are
  perceived as the most conducive to learning?
\item
  {[}Desription of the Learning (freetext fields, qualitative
  analysis){]}
\item
  RQ (Karasek, control for age, sex, jobscope + data from the FBS) - MLM

  Do stress and coping predict informal learning in teachers' daily work
  activities, as stated in Karasek's learning hypothesis?

  H: according to Karasek
\end{enumerate}

include a participation effect (control for a bias) H: higher
participation -\textgreater{} bigger pc\_learn Can time effects be found
in the data? Does continued experience sampling have an effect on
perceived informal learning?

\section{Research on Teachers' Workplace
Learning}\label{research-on-teachers-workplace-learning}

\subsection{Characteristics of the Teaching Profession and Vocational
Teachers' Daily Work
Activities}\label{characteristics-of-the-teaching-profession-and-vocational-teachers-daily-work-activities}

\subsubsection{Characteristics}\label{characteristics}

\begin{itemize}
\item
  High degree of freedom in the profession
\item
  subject (especially in vocational schools).

  \begin{itemize}
  \tightlist
  \item
    --\textgreater{} high demand to learn
  \end{itemize}
\item
  Needed: Alternative to Rothland (2013)???

  \begin{itemize}
  \tightlist
  \item
    Multiple sources that describe the characteristics
  \end{itemize}
\item
  teacher stressors and coping (briefly): outline from research on
  teacher stress

  \begin{itemize}
  \tightlist
  \item
    --\textgreater{} Lazarus \& Folkman (for Stress)
  \end{itemize}
\end{itemize}

\subsubsection{Teachers' Daily Work
Activities}\label{teachers-daily-work-activities}

\begin{itemize}
\tightlist
\item
  Overview teachers' work activities (framework from the literature)
\end{itemize}

\subsection{Workplace Learning and Vocational Teachers' Learning
Activities}\label{workplace-learning-and-vocational-teachers-learning-activities}

\begin{itemize}
\tightlist
\item
  formal, non-formal vs.~informal WPL (so far from bwpat!!!)
\item
  how do teachers learn?
\item
  include selected frameworks from literature (NK, MA)
\end{itemize}

In workplace learning, researchers typically differentiate between
formal, non-formal and informal learning (e.g., Coombs \& Ahmed, 1974;
Imants \& van Veen, 2010). Formal learning is typically defined as
structured learning in pedagogical settings such as university teacher
training. In these settings, learning occurs intentionally and is
planned (Marsick \& Watkins, 2015; UNESCO Institute for Statistics,
2012). Non-formal learning is ``institutionalized, intentional and
planned'' as well (UNESCO Institute for Statistics, 2012, p.~11) but in
contrast to formal learning, it is not part of the national
qualifications framework but includes training and development in
companies (Bilger et al., 2013, p.~20) such as information resources for
further teacher training. In contrast, informal learning is
unintentional and experiential. It occurs as a by-product of other
activities such as working (e.g., Marsick \& Watkins, 2015, p.~6; UNESCO
Institute for Statistics, 2012, p.~19). This learning is also referred
to as implicit learning (Eraut, 2004) or incidental learning (Marsick \&
Watkins, 2015). Though less conscious, this informal learning is
considered as a vital source of teachers' professional development. Work
task characteristics that foster informal workplace learning include
newness, complexity, collaboration and so forth (Hoekstra, 2007; Lohman,
2003; Rausch, 2013; Kwakman, 2003) many of which, as discussed above,
are also likely to cause stress (Karasek, 1979).

\begin{itemize}
\tightlist
\item
  Billett?!
\item
  Karasek

  \begin{itemize}
  \tightlist
  \item
    Studies on Karasek

    \begin{itemize}
    \tightlist
    \item
      The Job Demand-Control (-Support) Model and psychological
      well-being: A review of 20 years of empirical research
      (\url{https://doi.org/10.1080/026783799296084})
    \item
    \end{itemize}
  \end{itemize}
\end{itemize}

\begin{enumerate}
\def\labelenumi{\arabic{enumi}.}
\item
  Billett and others (which characteristics of the situation and the
  activity foster learning?!)
\item
  Consideration of stress (Karasek, \ldots)

  use stress as a characteristic of activities to introduce Karasek (and
  briefly talk about negative consequences: Lazarus \& Folkman)
\end{enumerate}

get concrete: Teachers' Learning Activities

\subsection{Learning Outomes from Vocational Teachers' Workplace
Learning}\label{learning-outomes-from-vocational-teachers-workplace-learning}

\begin{itemize}
\tightlist
\item
  Goal: categorizing learning outcomes that stem from teachers' informal
  workplace learning
\item
  For this, existing frameworks from workplace learning literature are
  analyzed and compared
\item
  Then, a category framework is developed deductively from existing
  frameworks and then adjusted inductively from the results from this
  study.
\end{itemize}

\subsubsection{Eraut (2004)}\label{eraut-2004}

8 categories:

\begin{itemize}
\tightlist
\item
\item
\end{itemize}

\subsubsection{Tynjälä (2013, 3-P
model)}\label{tynjuxe4luxe4-2013-3-p-model}

\subsubsection{Kyndt et al.~(2013)}\label{kyndt-et-al.-2013}

\subsubsection{Cerasoli et al.~(2018)}\label{cerasoli-et-al.-2018}

\subsubsection{Park (2020)}\label{park-2020}

\subsubsection{Smet et al.~(2022)}\label{smet-et-al.-2022}

--\textgreater{} also consider additional frameworks from teacher
professional development

\subsection{Job Demands and Job Resources in the Teaching
Profession}\label{job-demands-and-job-resources-in-the-teaching-profession}

\subsubsection{Job Demands / Stress}\label{job-demands-stress}

\subsubsection{Job Resources / Coping}\label{job-resources-coping}

\subsubsection{The Effect of Job Demands / Stress and Job Resources /
Coping on Vocational Teachers' Workplace
Learning}\label{the-effect-of-job-demands-stress-and-job-resources-coping-on-vocational-teachers-workplace-learning}

\section{Description of the Longitudinal
Study}\label{description-of-the-longitudinal-study}

\subsection{Research Design and
Sample}\label{research-design-and-sample}

This study is part of a research programme (AARL-BS) which was initiated
to investigate the relations between working hours, work activities, and
work experience, such as learning, stress and coping of teachers at
vocational schools. \ul{Data was collected in two studies, an online
survey study and an app-based diary study. This allowed for balancing
the advantages and disadvantages of the respective methods regarding the
estimation of working hours and the measuring of work experience, in
particular. Participation was voluntary and all participants provided
written informed consent.}

\subsubsection{Questionnaire Study}\label{questionnaire-study}

The survey study was conducted from February to November 2022. A sample
of 1,146 full-time teachers participated in the survey study, 74.3 \% of
which held no further management function beyond their teaching duties.
The mean age was 46.98 years and 39.1 \% of the sample is female. The
distribution of the survey sample is representative for vocational
teachers in the German federal state of Baden-Wuerttemberg with regard
to gender, age composition, level of employment, and administrative
district.

In the survey study, data on teachers' working time, the distribution of
the working hours between different tasks, working conditions, job
satisfaction, and further constructs were collected. The questionnaire
was developed on the basis of a comprehensive literature review (Aprea
\& Sarochan, 2023) and intensive consultations with representatives of
the Association of Vocational School Teachers in Baden-Württemberg
(BLV). \ul{See the other papers \ldots{}}

\subsubsection{Diary Study (go more into detail here, ESM,
design)}\label{diary-study-go-more-into-detail-here-esm-design}

The diary study took place from mid-March to mid-October 2022, including
weekends and vacation periods, excluding four weeks during the summer
holiday. A multi-cohort design was chosen to reduce participant burden.
Each of the five cohorts held the diary for one week and paused for four
weeks. The diary app was implemented using mQuest by the German online
service provider Cluetec (Karlsruhe). \ul{Diary entries from 145
full-time teachers were included, 75.2 \% of which without a management
function. The mean age is 44.99 years and 46.9 \% of the sample is
female. After intensive data preparation and filtering, the analysis is
based on 10.327 activities that were reported in the diary app}.

The participants were requested to record all work-related activities by
selecting the respective work activity from a given list of activities,
indicating start and end time and answering one item each for
experienced stress, coping, and learning related to the respective task.
\ul{In addition, in a weekly review, the participants were requested to
indicate the working hours for each day of the past week.} During a
cohort's diary period, three daily notifications reminded the
participants to record their work activities.

\subsection{Measures and Data
Analysis}\label{measures-and-data-analysis}

\begin{itemize}
\tightlist
\item
  Stress, Coping and Learning across the Daily Work Activities

  \begin{itemize}
  \tightlist
  \item
    Stress, coping and learning were all measured using 1 item scales
    self report
  \item
    Description of the Developed Task Framework
  \end{itemize}
\end{itemize}

From bwpat

Afterwards, the perception of stress, coping and learning at the given
work activity are evaluated by the participants. (1) Stress, (2) coping
and (3) learning are each designed with an 8-point Likert-scale with 0
as the lowest and 7 as the highest value. To avoid influencing entries
with a default value, ``-1'' is set as the default value in these three
items and must be changed to proceed. All three questions are depicted
with a slider to set the value and a brief explanation: (1) Did you find
this work activity stressful? (0 = not at all stressful; 7 = very
stressful; -1 = invalid entry); (2) How well were you able to cope with
this stress? (0 = not coped well at all; 7 = coped very well; -1 =
invalid entry); (3) Did you learn anything new for your job during this
work activity? (0 = learned nothing at all; 7 = learned very much; -1 =
invalid entry). Based on theoretical assumptions, participants could
only evaluate their coping for work activities with a stress-level above
0.

\subsubsection{Learning Outcome Framework (RQ
2)}\label{learning-outcome-framework-rq-2}

K. Kompetenzebene

KI. Individuum

Performanzebene

\subsection{Data Analysis (still from
bwpat)}\label{data-analysis-still-from-bwpat}

Descriptive statistics were calculated to address RQ1 to RQ3. Regarding
RQ4, a multiple linear regression analysis was conducted to investigate
the statistical prediction of job satisfaction based on working hours,
management function, stress, coping, and learning. Interaction terms
were checked. However, moderators showed no significant effects, so no
interactions were included in the final analysis.

\section{Results}\label{results}

\subsection{RQ1: Stress, Coping and Learning during Teachers' Daily Work
Activities}\label{rq1-stress-coping-and-learning-during-teachers-daily-work-activities}

\begin{itemize}
\tightlist
\item
  maybe too similar to bwpat article
\item
  may cause problems in comparison with bwpat paper (different sample
  size\ldots)
\item
  same approach as in bwpat paper: simple comparison of mean values and
  SDs across the activities
\item
  29 activities?! --\textgreater{} decision needed!
\end{itemize}

\begin{verbatim}
## Warning: There was 1 warning in `mutate()`.
## i In argument: `across(...)`.
## Caused by warning:
## ! Use of bare predicate functions was deprecated in tidyselect 1.1.0.
## i Please use wrap predicates in `where()` instead.
##   # Was:
##   data %>% select(is.numeric)
## 
##   # Now:
##   data %>% select(where(is.numeric))
\end{verbatim}

\begingroup\fontsize{9}{11}\selectfont

\begin{longtable}[t]{lccccccccc}
\caption{\label{tab:rq1 kable table}Descriptive Statistics regarding Stress, Coping and Learning during Teachers' Daily Work Activities}\\
\toprule
\multicolumn{1}{c}{Activity} & \multicolumn{3}{c}{Stress} & \multicolumn{3}{c}{Coping} & \multicolumn{3}{c}{Learning} \\
\cmidrule(l{3pt}r{3pt}){1-1} \cmidrule(l{3pt}r{3pt}){2-4} \cmidrule(l{3pt}r{3pt}){5-7} \cmidrule(l{3pt}r{3pt}){8-10}
 & M & SD & Mdn & M & SD & Mdn & M & SD & Mdn\\
\midrule
1 & 1.55 & 1.71 & 1.00 & 2.40 & 2.57 & 1.00 & 0.71 & 1.26 & 0.00\\
2 & 1.82 & 2.02 & 1.00 & 2.29 & 2.52 & 1.00 & 0.75 & 1.50 & 0.00\\
3 & 1.33 & 1.65 & 1.00 & 2.13 & 2.55 & 1.00 & 0.67 & 1.29 & 0.00\\
4 & 1.31 & 1.69 & 1.00 & 2.05 & 2.57 & 0.00 & 0.44 & 1.10 & 0.00\\
5 & 1.24 & 1.66 & 0.00 & 2.02 & 2.59 & 0.00 & 0.49 & 1.11 & 0.00\\
\addlinespace
6 & 0.97 & 1.51 & 0.00 & 1.77 & 2.52 & 0.00 & 0.82 & 1.48 & 0.00\\
7 & 1.15 & 1.58 & 0.00 & 1.97 & 2.58 & 0.00 & 0.90 & 1.55 & 0.00\\
8 & 0.80 & 1.27 & 0.00 & 1.76 & 2.56 & 0.00 & 0.42 & 1.00 & 0.00\\
9 & 0.93 & 1.53 & 0.00 & 1.57 & 2.48 & 0.00 & 0.20 & 0.77 & 0.00\\
10 & 0.32 & 0.97 & 0.00 & 0.59 & 1.68 & 0.00 & 0.24 & 0.93 & 0.00\\
\addlinespace
11 & 0.80 & 1.48 & 0.00 & 1.13 & 2.10 & 0.00 & 0.10 & 0.47 & 0.00\\
12 & 1.42 & 1.74 & 1.00 & 1.98 & 2.46 & 1.00 & 3.34 & 2.46 & 3.00\\
13 & 1.00 & 1.43 & 0.00 & 2.03 & 2.60 & 0.00 & 3.17 & 2.18 & 3.00\\
14 & 1.35 & 1.66 & 1.00 & 2.51 & 2.83 & 1.00 & 1.42 & 1.92 & 0.00\\
15 & 1.76 & 1.98 & 1.00 & 2.30 & 2.63 & 1.00 & 0.23 & 0.92 & 0.00\\
\addlinespace
16 & 1.33 & 1.68 & 1.00 & 2.18 & 2.56 & 0.00 & 1.41 & 1.80 & 1.00\\
17 & 1.41 & 1.75 & 1.00 & 1.77 & 2.31 & 0.00 & 0.88 & 1.47 & 0.00\\
18 & 1.64 & 1.74 & 1.00 & 2.66 & 2.64 & 2.00 & 1.69 & 2.07 & 1.00\\
19 & 1.84 & 1.92 & 1.00 & 2.38 & 2.54 & 1.00 & 0.85 & 1.55 & 0.00\\
20 & 0.79 & 1.11 & 0.00 & 2.10 & 2.69 & 0.00 & 0.72 & 1.36 & 0.00\\
\addlinespace
21 & 1.31 & 1.68 & 0.50 & 1.98 & 2.50 & 0.00 & 0.95 & 1.57 & 0.00\\
22 & 1.45 & 1.66 & 1.00 & 2.35 & 2.54 & 1.00 & 0.45 & 1.02 & 0.00\\
23 & 1.36 & 1.62 & 1.00 & 2.07 & 2.41 & 1.00 & 0.93 & 1.43 & 0.00\\
24 & 1.73 & 1.79 & 1.00 & 2.47 & 2.57 & 2.00 & 0.61 & 1.24 & 0.00\\
25 & 1.48 & 1.68 & 1.00 & 2.24 & 2.47 & 1.00 & 1.08 & 1.64 & 0.00\\
\addlinespace
26 & 1.64 & 1.89 & 1.00 & 2.68 & 2.72 & 2.00 & 0.78 & 1.32 & 0.00\\
27 & 1.61 & 1.84 & 1.00 & 2.17 & 2.40 & 1.00 & 0.99 & 1.46 & 0.00\\
28 & 1.29 & 1.55 & 1.00 & 2.27 & 2.56 & 1.00 & 1.21 & 1.78 & 0.00\\
29 & 1.83 & 2.12 & 1.00 & 2.07 & 2.42 & 1.00 & 0.93 & 1.76 & 0.00\\
\bottomrule
\end{longtable}
\endgroup{}

\subsection{RQ2: Description of the Learning During Teachers' Daily Work
Activities}\label{rq2-description-of-the-learning-during-teachers-daily-work-activities}

\begin{longtable}[]{@{}
  >{\raggedleft\arraybackslash}p{(\columnwidth - 58\tabcolsep) * \real{0.0084}}
  >{\raggedleft\arraybackslash}p{(\columnwidth - 58\tabcolsep) * \real{0.0517}}
  >{\raggedleft\arraybackslash}p{(\columnwidth - 58\tabcolsep) * \real{0.0529}}
  >{\raggedleft\arraybackslash}p{(\columnwidth - 58\tabcolsep) * \real{0.0818}}
  >{\raggedleft\arraybackslash}p{(\columnwidth - 58\tabcolsep) * \real{0.0277}}
  >{\raggedleft\arraybackslash}p{(\columnwidth - 58\tabcolsep) * \real{0.0229}}
  >{\raggedleft\arraybackslash}p{(\columnwidth - 58\tabcolsep) * \real{0.0361}}
  >{\raggedleft\arraybackslash}p{(\columnwidth - 58\tabcolsep) * \real{0.0361}}
  >{\raggedleft\arraybackslash}p{(\columnwidth - 58\tabcolsep) * \real{0.0361}}
  >{\raggedleft\arraybackslash}p{(\columnwidth - 58\tabcolsep) * \real{0.0325}}
  >{\raggedleft\arraybackslash}p{(\columnwidth - 58\tabcolsep) * \real{0.0156}}
  >{\raggedleft\arraybackslash}p{(\columnwidth - 58\tabcolsep) * \real{0.0445}}
  >{\raggedleft\arraybackslash}p{(\columnwidth - 58\tabcolsep) * \real{0.0457}}
  >{\raggedleft\arraybackslash}p{(\columnwidth - 58\tabcolsep) * \real{0.0385}}
  >{\raggedleft\arraybackslash}p{(\columnwidth - 58\tabcolsep) * \real{0.0205}}
  >{\raggedleft\arraybackslash}p{(\columnwidth - 58\tabcolsep) * \real{0.0349}}
  >{\raggedleft\arraybackslash}p{(\columnwidth - 58\tabcolsep) * \real{0.0132}}
  >{\raggedleft\arraybackslash}p{(\columnwidth - 58\tabcolsep) * \real{0.0084}}
  >{\raggedleft\arraybackslash}p{(\columnwidth - 58\tabcolsep) * \real{0.0505}}
  >{\raggedleft\arraybackslash}p{(\columnwidth - 58\tabcolsep) * \real{0.0433}}
  >{\raggedleft\arraybackslash}p{(\columnwidth - 58\tabcolsep) * \real{0.0253}}
  >{\raggedleft\arraybackslash}p{(\columnwidth - 58\tabcolsep) * \real{0.0120}}
  >{\raggedleft\arraybackslash}p{(\columnwidth - 58\tabcolsep) * \real{0.0313}}
  >{\raggedleft\arraybackslash}p{(\columnwidth - 58\tabcolsep) * \real{0.0229}}
  >{\raggedleft\arraybackslash}p{(\columnwidth - 58\tabcolsep) * \real{0.0156}}
  >{\raggedleft\arraybackslash}p{(\columnwidth - 58\tabcolsep) * \real{0.0554}}
  >{\raggedleft\arraybackslash}p{(\columnwidth - 58\tabcolsep) * \real{0.0313}}
  >{\raggedleft\arraybackslash}p{(\columnwidth - 58\tabcolsep) * \real{0.0433}}
  >{\raggedleft\arraybackslash}p{(\columnwidth - 58\tabcolsep) * \real{0.0289}}
  >{\raggedleft\arraybackslash}p{(\columnwidth - 58\tabcolsep) * \real{0.0325}}@{}}
\caption{Learning Categories across Vocational Teachers' Daily Work
Activities (RQ2)}\tabularnewline
\toprule\noalign{}
\begin{minipage}[b]{\linewidth}\raggedleft
act\_no
\end{minipage} & \begin{minipage}[b]{\linewidth}\raggedleft
Bewusstsein der Bedeutung von Gelassenheit
\end{minipage} & \begin{minipage}[b]{\linewidth}\raggedleft
Bewusstsein der Bedeutung von Konzentration
\end{minipage} & \begin{minipage}[b]{\linewidth}\raggedleft
Bewusstsein der Bedeutung von gewissenhaftem, sorgfältigen Arbeiten
\end{minipage} & \begin{minipage}[b]{\linewidth}\raggedleft
Emotionale Erschöpfung
\end{minipage} & \begin{minipage}[b]{\linewidth}\raggedleft
Emotionsregulation
\end{minipage} & \begin{minipage}[b]{\linewidth}\raggedleft
Festigung der eigenen Routine
\end{minipage} & \begin{minipage}[b]{\linewidth}\raggedleft
Festigung von Kernkompetenzen
\end{minipage} & \begin{minipage}[b]{\linewidth}\raggedleft
Fähigkeit zur Selbstreflexion
\end{minipage} & \begin{minipage}[b]{\linewidth}\raggedleft
Gegenseitige Unterstützung
\end{minipage} & \begin{minipage}[b]{\linewidth}\raggedleft
Gelassenheit
\end{minipage} & \begin{minipage}[b]{\linewidth}\raggedleft
Gemeinsame Planung und Problemlösung
\end{minipage} & \begin{minipage}[b]{\linewidth}\raggedleft
Gewissenhaftes, sorgfältiges Arbeiten
\end{minipage} & \begin{minipage}[b]{\linewidth}\raggedleft
Jobunzufriedenheit/Frustration
\end{minipage} & \begin{minipage}[b]{\linewidth}\raggedleft
Jobzufriedenheit
\end{minipage} & \begin{minipage}[b]{\linewidth}\raggedleft
Lernen neuer Arbeitsmethoden
\end{minipage} & \begin{minipage}[b]{\linewidth}\raggedleft
Motivation
\end{minipage} & \begin{minipage}[b]{\linewidth}\raggedleft
Nichts
\end{minipage} & \begin{minipage}[b]{\linewidth}\raggedleft
Pausenbewusstsein und Ressourcenschonung
\end{minipage} & \begin{minipage}[b]{\linewidth}\raggedleft
Positive Einstellung zur Teamarbeit
\end{minipage} & \begin{minipage}[b]{\linewidth}\raggedleft
Situationsdiagnostik
\end{minipage} & \begin{minipage}[b]{\linewidth}\raggedleft
Sonstiges
\end{minipage} & \begin{minipage}[b]{\linewidth}\raggedleft
Vermutliche Falscheingabe
\end{minipage} & \begin{minipage}[b]{\linewidth}\raggedleft
Wissenserweiterung
\end{minipage} & \begin{minipage}[b]{\linewidth}\raggedleft
Effektivität
\end{minipage} & \begin{minipage}[b]{\linewidth}\raggedleft
Bewusstsein der Bedeutung von Selbstreflexion
\end{minipage} & \begin{minipage}[b]{\linewidth}\raggedleft
Psychologische Sicherheit
\end{minipage} & \begin{minipage}[b]{\linewidth}\raggedleft
positive Einstellung zur Teamarbeit
\end{minipage} & \begin{minipage}[b]{\linewidth}\raggedleft
Reduzierung von Fehlern
\end{minipage} & \begin{minipage}[b]{\linewidth}\raggedleft
Effektivität auf Teamebene
\end{minipage} \\
\midrule\noalign{}
\endfirsthead
\toprule\noalign{}
\begin{minipage}[b]{\linewidth}\raggedleft
act\_no
\end{minipage} & \begin{minipage}[b]{\linewidth}\raggedleft
Bewusstsein der Bedeutung von Gelassenheit
\end{minipage} & \begin{minipage}[b]{\linewidth}\raggedleft
Bewusstsein der Bedeutung von Konzentration
\end{minipage} & \begin{minipage}[b]{\linewidth}\raggedleft
Bewusstsein der Bedeutung von gewissenhaftem, sorgfältigen Arbeiten
\end{minipage} & \begin{minipage}[b]{\linewidth}\raggedleft
Emotionale Erschöpfung
\end{minipage} & \begin{minipage}[b]{\linewidth}\raggedleft
Emotionsregulation
\end{minipage} & \begin{minipage}[b]{\linewidth}\raggedleft
Festigung der eigenen Routine
\end{minipage} & \begin{minipage}[b]{\linewidth}\raggedleft
Festigung von Kernkompetenzen
\end{minipage} & \begin{minipage}[b]{\linewidth}\raggedleft
Fähigkeit zur Selbstreflexion
\end{minipage} & \begin{minipage}[b]{\linewidth}\raggedleft
Gegenseitige Unterstützung
\end{minipage} & \begin{minipage}[b]{\linewidth}\raggedleft
Gelassenheit
\end{minipage} & \begin{minipage}[b]{\linewidth}\raggedleft
Gemeinsame Planung und Problemlösung
\end{minipage} & \begin{minipage}[b]{\linewidth}\raggedleft
Gewissenhaftes, sorgfältiges Arbeiten
\end{minipage} & \begin{minipage}[b]{\linewidth}\raggedleft
Jobunzufriedenheit/Frustration
\end{minipage} & \begin{minipage}[b]{\linewidth}\raggedleft
Jobzufriedenheit
\end{minipage} & \begin{minipage}[b]{\linewidth}\raggedleft
Lernen neuer Arbeitsmethoden
\end{minipage} & \begin{minipage}[b]{\linewidth}\raggedleft
Motivation
\end{minipage} & \begin{minipage}[b]{\linewidth}\raggedleft
Nichts
\end{minipage} & \begin{minipage}[b]{\linewidth}\raggedleft
Pausenbewusstsein und Ressourcenschonung
\end{minipage} & \begin{minipage}[b]{\linewidth}\raggedleft
Positive Einstellung zur Teamarbeit
\end{minipage} & \begin{minipage}[b]{\linewidth}\raggedleft
Situationsdiagnostik
\end{minipage} & \begin{minipage}[b]{\linewidth}\raggedleft
Sonstiges
\end{minipage} & \begin{minipage}[b]{\linewidth}\raggedleft
Vermutliche Falscheingabe
\end{minipage} & \begin{minipage}[b]{\linewidth}\raggedleft
Wissenserweiterung
\end{minipage} & \begin{minipage}[b]{\linewidth}\raggedleft
Effektivität
\end{minipage} & \begin{minipage}[b]{\linewidth}\raggedleft
Bewusstsein der Bedeutung von Selbstreflexion
\end{minipage} & \begin{minipage}[b]{\linewidth}\raggedleft
Psychologische Sicherheit
\end{minipage} & \begin{minipage}[b]{\linewidth}\raggedleft
positive Einstellung zur Teamarbeit
\end{minipage} & \begin{minipage}[b]{\linewidth}\raggedleft
Reduzierung von Fehlern
\end{minipage} & \begin{minipage}[b]{\linewidth}\raggedleft
Effektivität auf Teamebene
\end{minipage} \\
\midrule\noalign{}
\endhead
\bottomrule\noalign{}
\endlastfoot
1 & 4 & 1 & 11 & 5 & 2 & 1 & 22 & 20 & 1 & 9 & 1 & 10 & 2 & 5 & 8 & 2 &
314 & 2 & 8 & 288 & 4 & 1 & 51 & 0 & 0 & 0 & 0 & 0 & 0 \\
2 & 0 & 0 & 0 & 0 & 0 & 0 & 0 & 0 & 0 & 0 & 0 & 0 & 0 & 0 & 0 & 0 & 21 &
0 & 0 & 9 & 1 & 0 & 3 & 0 & 0 & 0 & 0 & 0 & 0 \\
3 & 1 & 1 & 18 & 3 & 0 & 4 & 22 & 18 & 1 & 2 & 0 & 17 & 5 & 1 & 6 & 0 &
397 & 3 & 4 & 141 & 3 & 9 & 138 & 2 & 0 & 0 & 0 & 0 & 0 \\
4 & 0 & 0 & 8 & 1 & 0 & 0 & 6 & 3 & 0 & 0 & 0 & 2 & 2 & 0 & 0 & 0 & 100
& 0 & 2 & 24 & 1 & 1 & 12 & 0 & 0 & 0 & 0 & 0 & 0 \\
5 & 0 & 0 & 1 & 1 & 0 & 0 & 1 & 4 & 0 & 0 & 0 & 4 & 0 & 1 & 1 & 0 & 83 &
0 & 2 & 48 & 2 & 0 & 15 & 0 & 0 & 0 & 0 & 0 & 0 \\
6 & 0 & 1 & 6 & 0 & 0 & 0 & 7 & 10 & 4 & 3 & 5 & 2 & 0 & 0 & 4 & 0 & 243
& 4 & 29 & 63 & 5 & 1 & 75 & 0 & 2 & 6 & 0 & 0 & 0 \\
7 & 0 & 0 & 1 & 0 & 0 & 0 & 0 & 1 & 0 & 0 & 0 & 0 & 0 & 0 & 0 & 0 & 44 &
0 & 7 & 12 & 1 & 1 & 18 & 0 & 0 & 0 & 1 & 0 & 0 \\
8 & 0 & 1 & 12 & 1 & 0 & 2 & 2 & 3 & 0 & 0 & 0 & 13 & 1 & 0 & 0 & 0 & 99
& 0 & 1 & 12 & 0 & 2 & 18 & 0 & 0 & 0 & 0 & 0 & 0 \\
9 & 0 & 1 & 1 & 0 & 0 & 0 & 0 & 0 & 0 & 0 & 0 & 0 & 0 & 0 & 0 & 0 & 28 &
0 & 0 & 5 & 0 & 0 & 2 & 0 & 0 & 0 & 0 & 0 & 0 \\
10 & 1 & 0 & 1 & 1 & 0 & 0 & 0 & 1 & 0 & 0 & 0 & 0 & 3 & 0 & 0 & 0 & 78
& 13 & 2 & 9 & 0 & 1 & 5 & 0 & 0 & 0 & 0 & 0 & 0 \\
11 & 0 & 0 & 0 & 0 & 0 & 0 & 1 & 1 & 0 & 0 & 0 & 2 & 0 & 0 & 0 & 0 & 16
& 0 & 0 & 0 & 0 & 0 & 1 & 0 & 0 & 0 & 0 & 0 & 0 \\
12 & 0 & 1 & 0 & 0 & 0 & 0 & 9 & 4 & 0 & 2 & 0 & 0 & 0 & 0 & 4 & 0 & 50
& 0 & 5 & 21 & 5 & 0 & 57 & 0 & 0 & 0 & 0 & 0 & 0 \\
13 & 1 & 0 & 0 & 0 & 1 & 0 & 1 & 2 & 0 & 0 & 2 & 0 & 0 & 0 & 1 & 0 & 30
& 0 & 0 & 7 & 0 & 0 & 60 & 0 & 0 & 0 & 0 & 0 & 0 \\
14 & 0 & 0 & 1 & 0 & 0 & 0 & 0 & 0 & 0 & 1 & 0 & 0 & 0 & 0 & 0 & 0 & 32
& 0 & 2 & 4 & 1 & 0 & 3 & 0 & 0 & 0 & 0 & 0 & 0 \\
15 & 0 & 0 & 0 & 0 & 0 & 5 & 0 & 0 & 0 & 1 & 0 & 0 & 1 & 0 & 0 & 0 & 30
& 0 & 0 & 2 & 0 & 0 & 2 & 0 & 0 & 0 & 0 & 0 & 0 \\
16 & 0 & 0 & 1 & 0 & 1 & 0 & 11 & 2 & 0 & 0 & 1 & 2 & 1 & 0 & 0 & 0 & 49
& 0 & 14 & 5 & 0 & 1 & 25 & 0 & 0 & 1 & 0 & 0 & 0 \\
17 & 0 & 0 & 1 & 0 & 0 & 0 & 0 & 1 & 2 & 0 & 0 & 0 & 0 & 0 & 0 & 0 & 20
& 0 & 1 & 7 & 0 & 0 & 17 & 0 & 0 & 0 & 0 & 0 & 0 \\
18 & 0 & 0 & 0 & 0 & 0 & 0 & 1 & 1 & 0 & 1 & 3 & 1 & 1 & 0 & 0 & 0 & 37
& 1 & 4 & 9 & 1 & 1 & 16 & 0 & 1 & 1 & 0 & 1 & 0 \\
19 & 0 & 2 & 3 & 0 & 0 & 0 & 5 & 3 & 1 & 0 & 2 & 1 & 4 & 1 & 0 & 0 & 111
& 0 & 5 & 37 & 3 & 0 & 28 & 0 & 0 & 0 & 0 & 0 & 0 \\
20 & 0 & 0 & 0 & 0 & 0 & 0 & 1 & 0 & 0 & 0 & 0 & 0 & 0 & 0 & 0 & 0 & 2 &
0 & 0 & 0 & 0 & 0 & 1 & 0 & 0 & 0 & 0 & 0 & 0 \\
21 & 0 & 1 & 3 & 0 & 1 & 0 & 8 & 1 & 0 & 0 & 0 & 1 & 1 & 0 & 2 & 0 & 70
& 1 & 7 & 31 & 1 & 1 & 29 & 0 & 1 & 0 & 0 & 0 & 0 \\
22 & 1 & 0 & 0 & 0 & 0 & 0 & 1 & 1 & 0 & 1 & 3 & 1 & 1 & 1 & 0 & 0 & 45
& 0 & 5 & 9 & 0 & 0 & 6 & 0 & 1 & 0 & 0 & 0 & 1 \\
23 & 0 & 0 & 0 & 0 & 0 & 0 & 3 & 0 & 0 & 0 & 0 & 0 & 0 & 0 & 1 & 0 & 23
& 0 & 1 & 5 & 0 & 0 & 3 & 0 & 0 & 0 & 0 & 0 & 0 \\
24 & 0 & 1 & 2 & 0 & 0 & 0 & 0 & 1 & 0 & 1 & 0 & 2 & 0 & 0 & 0 & 0 & 31
& 0 & 1 & 7 & 0 & 0 & 3 & 0 & 0 & 0 & 0 & 0 & 0 \\
25 & 0 & 0 & 0 & 0 & 0 & 0 & 3 & 2 & 0 & 0 & 1 & 2 & 1 & 0 & 0 & 0 & 45
& 0 & 3 & 13 & 2 & 0 & 9 & 0 & 1 & 0 & 0 & 0 & 0 \\
26 & 0 & 0 & 1 & 0 & 0 & 0 & 1 & 0 & 0 & 0 & 0 & 0 & 0 & 0 & 0 & 0 & 32
& 0 & 1 & 15 & 0 & 0 & 10 & 0 & 0 & 0 & 0 & 0 & 0 \\
27 & 0 & 0 & 3 & 0 & 0 & 0 & 0 & 0 & 0 & 0 & 0 & 0 & 0 & 0 & 0 & 0 & 8 &
0 & 1 & 3 & 1 & 1 & 2 & 0 & 0 & 0 & 0 & 0 & 0 \\
28 & 0 & 0 & 2 & 0 & 0 & 0 & 2 & 0 & 0 & 1 & 0 & 2 & 1 & 0 & 0 & 0 & 44
& 0 & 5 & 8 & 1 & 0 & 9 & 0 & 0 & 0 & 0 & 0 & 0 \\
29 & 1 & 0 & 2 & 1 & 0 & 0 & 0 & 2 & 0 & 0 & 0 & 1 & 0 & 0 & 0 & 0 & 13
& 1 & 2 & 9 & 2 & 0 & 6 & 0 & 0 & 1 & 0 & 0 & 0 \\
\end{longtable}

\subsection{RQ3: Predicting Vocational Teachers' Informal Learning Using
Stress and Coping during their Daily Work Activities (as stated in
Karasek's Learning
Hypothesis)}\label{rq3-predicting-vocational-teachers-informal-learning-using-stress-and-coping-during-their-daily-work-activities-as-stated-in-karaseks-learning-hypothesis}

\begin{itemize}
\tightlist
\item
  interaction effect
\end{itemize}

\begin{longtable}[]{@{}llllllllll@{}}
\caption{Correlations between Variables RQ3}\tabularnewline
\toprule\noalign{}
Measure & 1 & 2 & 3 & 4 & 5 & 6 & 7 & 8 & 9 \\
\midrule\noalign{}
\endfirsthead
\toprule\noalign{}
Measure & 1 & 2 & 3 & 4 & 5 & 6 & 7 & 8 & 9 \\
\midrule\noalign{}
\endhead
\bottomrule\noalign{}
\endlastfoot
1. sex & --- & --- & --- & --- & --- & --- & --- & --- & --- \\
2. age & .12 & --- & --- & --- & --- & --- & --- & --- & --- \\
3. jobscope & .31 & -.06 & --- & --- & --- & --- & --- & --- & --- \\
4. stress & .04 & -.02 & -.04 & --- & --- & --- & --- & --- & --- \\
5. coping & .03 & -.00 & -.03 & .51 & --- & --- & --- & --- & --- \\
6. pc\_learn & .03 & -.01 & -.03 & .26 & .23 & --- & --- & --- & --- \\
7. n\_entry & .01 & .01 & .12 & -.07 & -.00 & -.07 & --- & --- & --- \\
8. stress\_z & .04 & -.02 & -.04 & 1.00 & .51 & .26 & -.07 & --- &
--- \\
9. coping\_z & .03 & -.00 & -.03 & .51 & 1.00 & .23 & -.00 & .51 &
--- \\
\end{longtable}

\begin{longtable}[]{@{}lllr@{}}
\caption{ICC RQ3 (REML)}\tabularnewline
\toprule\noalign{}
ICC\_adjusted & ICC\_conditional & ICC\_unadjusted & Freq \\
\midrule\noalign{}
\endfirsthead
\toprule\noalign{}
ICC\_adjusted & ICC\_conditional & ICC\_unadjusted & Freq \\
\midrule\noalign{}
\endhead
\bottomrule\noalign{}
\endlastfoot
0.2865 & 0.2865 & 0.2865 & 1 \\
\end{longtable}

\begin{table}[H]
\centering\centering
\caption{\label{tab:rq3 results table}Summary Multilevel Model RQ3}
\centering
\begin{tabular}[t]{lcc}
\toprule
  & Model 1 & Model 2\\
\midrule
(Intercept) & \num{0.829}*** & \num{1.645}***\\
 & (\num{0.037}) & (\num{0.267})\\
 & (\num{<0.001}) & (\num{<0.001})\\
 & {}[\num{0.757}, \num{0.901}] & {}[\num{1.122}, \num{2.169}]\\
sex &  & \num{0.073}\\
 &  & \vphantom{2} (\num{0.071})\\
 &  & (\num{0.299})\\
 &  & {}[\num{-0.065}, \num{0.212}]\\
age &  & \num{-0.007}*\\
 &  & \vphantom{1} (\num{0.003})\\
 &  & (\num{0.034})\\
 &  & {}[\num{-0.014}, \num{-0.001}]\\
jobscope &  & \num{-0.004}\\
 &  & \vphantom{1} (\num{0.002})\\
 &  & (\num{0.111})\\
 &  & {}[\num{-0.008}, \num{0.001}]\\
stress\_z &  & \num{0.171}***\\
 &  & \vphantom{1} (\num{0.007})\\
 &  & \vphantom{18} (\num{<0.001})\\
 &  & {}[\num{0.156}, \num{0.185}]\\
coping\_z &  & \num{0.079}***\\
 &  & (\num{0.007})\\
 &  & \vphantom{17} (\num{<0.001})\\
 &  & {}[\num{0.065}, \num{0.094}]\\
n\_entry &  & \num{0.000}\\
 &  & (\num{0.000})\\
 &  & (\num{0.113})\\
 &  & {}[\num{0.000}, \num{0.000}]\\
act\_no\_1 &  & \num{-0.312}***\\
 &  & \vphantom{1} (\num{0.067})\\
 &  & \vphantom{16} (\num{<0.001})\\
 &  & {}[\num{-0.443}, \num{-0.180}]\\
act\_no\_2 &  & \num{-0.311}**\\
 &  & (\num{0.097})\\
 &  & (\num{0.001})\\
 &  & {}[\num{-0.501}, \num{-0.121}]\\
act\_no\_3 &  & \num{-0.337}***\\
 &  & (\num{0.067})\\
 &  & \vphantom{15} (\num{<0.001})\\
 &  & {}[\num{-0.468}, \num{-0.205}]\\
act\_no\_4 &  & \num{-0.508}***\\
 &  & \vphantom{1} (\num{0.071})\\
 &  & \vphantom{14} (\num{<0.001})\\
 &  & {}[\num{-0.648}, \num{-0.369}]\\
act\_no\_5 &  & \num{-0.414}***\\
 &  & (\num{0.072})\\
 &  & \vphantom{13} (\num{<0.001})\\
 &  & {}[\num{-0.555}, \num{-0.274}]\\
act\_no\_6 &  & \num{-0.126}+\\
 &  & (\num{0.068})\\
 &  & (\num{0.065})\\
 &  & {}[\num{-0.259}, \num{0.008}]\\
act\_no\_7 &  & \num{-0.174}*\\
 &  & (\num{0.083})\\
 &  & (\num{0.036})\\
 &  & {}[\num{-0.337}, \num{-0.012}]\\
act\_no\_8 &  & \num{-0.487}***\\
 &  & \vphantom{1} (\num{0.069})\\
 &  & \vphantom{12} (\num{<0.001})\\
 &  & {}[\num{-0.623}, \num{-0.351}]\\
act\_no\_9 &  & \num{-0.760}***\\
 &  & \vphantom{1} (\num{0.081})\\
 &  & \vphantom{11} (\num{<0.001})\\
 &  & {}[\num{-0.919}, \num{-0.600}]\\
act\_no\_10 &  & \num{-0.556}***\\
 &  & (\num{0.069})\\
 &  & \vphantom{10} (\num{<0.001})\\
 &  & {}[\num{-0.692}, \num{-0.420}]\\
act\_no\_11 &  & \num{-0.769}***\\
 &  & (\num{0.080})\\
 &  & \vphantom{9} (\num{<0.001})\\
 &  & {}[\num{-0.927}, \num{-0.612}]\\
act\_no\_12 &  & \num{2.279}***\\
 &  & \vphantom{1} (\num{0.086})\\
 &  & \vphantom{8} (\num{<0.001})\\
 &  & {}[\num{2.109}, \num{2.448}]\\
act\_no\_13 &  & \num{2.203}***\\
 &  & (\num{0.092})\\
 &  & \vphantom{7} (\num{<0.001})\\
 &  & {}[\num{2.023}, \num{2.382}]\\
act\_no\_14 &  & \num{0.351}***\\
 &  & (\num{0.107})\\
 &  & \vphantom{6} (\num{<0.001})\\
 &  & {}[\num{0.142}, \num{0.560}]\\
act\_no\_15 &  & \num{-0.761}***\\
 &  & (\num{0.094})\\
 &  & \vphantom{5} (\num{<0.001})\\
 &  & {}[\num{-0.945}, \num{-0.577}]\\
act\_no\_16 &  & \num{0.306}***\\
 &  & (\num{0.079})\\
 &  & \vphantom{4} (\num{<0.001})\\
 &  & {}[\num{0.150}, \num{0.461}]\\
act\_no\_17 &  & \num{-0.012}\\
 &  & (\num{0.090})\\
 &  & (\num{0.894})\\
 &  & {}[\num{-0.189}, \num{0.165}]\\
act\_no\_18 &  & \num{0.437}***\\
 &  & (\num{0.088})\\
 &  & \vphantom{3} (\num{<0.001})\\
 &  & {}[\num{0.264}, \num{0.610}]\\
act\_no\_19 &  & \num{-0.211}**\\
 &  & (\num{0.071})\\
 &  & (\num{0.003})\\
 &  & {}[\num{-0.350}, \num{-0.072}]\\
act\_no\_20 &  & \num{-0.679}**\\
 &  & (\num{0.224})\\
 &  & (\num{0.002})\\
 &  & {}[\num{-1.117}, \num{-0.241}]\\
act\_no\_21 &  & \num{0.019}\\
 &  & (\num{0.075})\\
 &  & (\num{0.794})\\
 &  & {}[\num{-0.127}, \num{0.166}]\\
act\_no\_22 &  & \num{-0.390}***\\
 &  & (\num{0.076})\\
 &  & \vphantom{2} (\num{<0.001})\\
 &  & {}[\num{-0.538}, \num{-0.242}]\\
act\_no\_23 &  & \num{0.070}\\
 &  & (\num{0.089})\\
 &  & (\num{0.434})\\
 &  & {}[\num{-0.105}, \num{0.244}]\\
act\_no\_24 &  & \num{-0.302}***\\
 &  & (\num{0.084})\\
 &  & \vphantom{1} (\num{<0.001})\\
 &  & {}[\num{-0.467}, \num{-0.137}]\\
act\_no\_25 &  & \num{0.018}\\
 &  & (\num{0.081})\\
 &  & (\num{0.820})\\
 &  & {}[\num{-0.140}, \num{0.177}]\\
act\_no\_26 &  & \num{-0.218}*\\
 &  & (\num{0.086})\\
 &  & (\num{0.011})\\
 &  & {}[\num{-0.387}, \num{-0.050}]\\
act\_no\_27 &  & \num{-0.114}\\
 &  & (\num{0.124})\\
 &  & (\num{0.358})\\
 &  & {}[\num{-0.357}, \num{0.129}]\\
act\_no\_28 &  & \num{0.229}**\\
 &  & (\num{0.087})\\
 &  & (\num{0.009})\\
 &  & {}[\num{0.058}, \num{0.401}]\\
stress\_z × coping\_z &  & \num{0.094}***\\
 &  & (\num{0.008})\\
 &  & (\num{<0.001})\\
 &  & {}[\num{0.077}, \num{0.110}]\\
SD (Intercept code) & \num{0.759} & \num{0.665}\\
SD (Observations) & \num{1.199} & \num{1.115}\\
\midrule
Num.Obs. & \num{41598} & \num{41598}\\
R2 Marg. & \num{0.000} & \num{0.120}\\
R2 Cond. & \num{0.286} & \num{0.351}\\
AIC & \num{134670.8} & \num{128805.2}\\
BIC & \num{134696.7} & \num{129133.4}\\
RMSE & \num{1.19} & \num{1.11}\\
\bottomrule
\multicolumn{3}{l}{\rule{0pt}{1em}+ p \textbackslash{}num\{< 0.1\}, * p \textbackslash{}num\{< 0.05\}, ** p \textbackslash{}num\{< 0.01\}, *** p \textbackslash{}num\{< 0.001\}}\\
\end{tabular}
\end{table}

\section{Conclusion}\label{conclusion}

\newpage

\section{Data availability statement}\label{data-availability-statement}

The anonymized data are available on Mendeley Data
(\url{https://www.elsevier.com/researcher/author/tools-and-resources/research-data})
under the following link: \ldots{}\\

\section{References}\label{references}

\phantomsection\label{refs}
\begin{CSLReferences}{1}{0}
\bibitem[\citeproctext]{ref-rauschUsingDiariesResearch2014}
Rausch, A. (2014). Using {Diaries} in {Research} on {Work} and
{Learning}. In C. Harteis, A. Rausch, \& J. Seifried (Eds.),
\emph{Discourses on {Professional Learning}} (Vol. 9, pp. 341--366).
Springer Netherlands.
\href{https://doi.org/10.1007/978-94-007-7012-6‗\%2017}{https://doi.org/10.1007/978-94-007-7012-6‗
17}

\end{CSLReferences}

\newpage

\section{Appendix}\label{appendix}

\subsection*{Possible Journals}\label{possible-journals}
\addcontentsline{toc}{subsection}{Possible Journals}

Teaching and Teacher Education (IF: 4.0),\\
\url{https://doi.org/10.1016/S0742-051X(02)00101-4} was also published
here

Regelungen/Hinweise:

\begin{itemize}
\item
  Abstract: 100 words, 3 - 6 keywords
\item
  Report: 5.000-9.000 words
\end{itemize}

Ansonsten:

\begin{itemize}
\item
  Human Resource Development International (IF: 3.8)

  \begin{itemize}
  \tightlist
  \item
    \url{https://doi.org/10.1080/13678860010004123} was also published
    here
  \end{itemize}
\item
  Journal of Workplace Learning (IF: )
\item
  Vocations and Learning (IF: 1.9)
\item
  Learning Environments Research (IF: 2.7)
\item
  Technology, Knowledge and Learning (IF: 3.0) - not that
  fitting\ldots{}
\item
  Empirical Research in Vocational Education and Training (IF: 1.6)
\item
  Learning and Instruction (IF: 4.7) - not that fitting\ldots{}
\end{itemize}

\end{document}
